\documentclass[11pt]{article}
\usepackage[margin=1in]{geometry}
\usepackage{amsmath}
\usepackage{enumitem}
\usepackage{algorithmicx}
\usepackage{listings}

\title{CSE - 015: Homework 5}
\author{Jaime Rivera}

\begin{document}
\maketitle

\section{Complexity Analysis}

Derive a complexity function for the following algorithms:

\begin{lstlisting}

def doNothing(someList):
       return False

\end{lstlisting}


\begin{itemize}

\item Answer: O(1)

\end{itemize} 


\begin{lstlisting}

def doSomething(someList):
       if len(someList) = = 0:
          return 0
       else if len(list = = 1):
          return 1
       else:
          return doSomething(someList[1:])

\end{lstlisting}


\begin{itemize}

\item Answer: O(n)

\end{itemize} 



\begin{lstlisting}

def doSomethingElse(someList):
       n = len(someList)
       for i in range (n):
             for j in range(n):
                  if someList[i] > someList[j]
                     temp = someList[i]
                     someList[i] = someList[j]
                     someList[j] = temp
           return someList

\end{lstlisting}


\begin{itemize}

\item Answer: $O(n^{2})$

\end{itemize} 


\section{Order of Complexity}

Prove the following:

f(n) = O(g(n)) if there exists a positive integer $n_0$ and a positive constant c, such that f(n) $<=$ $c*g(n) \bigvee$ n $>=$ $n_0$

\begin{itemize}

\item $f(n) = 3n + 2 $ $ \in O(n) $

\begin{itemize}

\item Answer: Clearly, for c = 4 and $n >= 2$, we have
\newline 0 $<=$ f(n) $<=$ 4n
\newline Hence, f(n) = O(n)

\end{itemize} 

\end{itemize}


\begin{itemize}

\item $g(n) = 7  \in O(1) $

\begin{itemize}

\item Answer: Clearly, for c = 8 and n $>=$ 0, we have 
\newline 0 $<=$ f(n) $<=$ 8*1
\newline Hence, f(n) = O(1)

\end{itemize} 


\end{itemize}


\begin{itemize}

\item $h(n) = n^{2} + 2n + 4 \in O(n^{2})$

\begin{itemize}

\item Answer: Clearly, for c = 3 and n $>=$ 3, we have
\newline 0 $<=$ f(n) $<=$ $3*n^{2}$
\newline Hence, $f(n) = O(n^{2})$

\end{itemize} 

\end{itemize}


\section{Mathematical Induction}

Use mathematical induction to show that the following results hold for all positive integers:



\begin{itemize}

\item $ 1 + 2 + 3 + ... + n = \frac{n(n+1)} {2} $

\begin{itemize}

\item Answer: 

for n = 1:

LHS = 1
\newline RHS =$ \frac{1(1+1)} {2}= \frac{2}{2} = 1 $ 
\newline so, LHS = RHS for n = 1
\newline lets assume LHS = RHS for n = k
\newline so, $ 1+ 2 +...+ k $ = $ \frac{k(k+1)}{2} $
\newline for n = k + 1:
\newline LHS = 1+2+ ...+k + (k+1)
\newline =$\frac{k(k+1)}{2} + (k+1)$
\newline =$k(k+1) + \frac {2(k+1)}{2}$
\newline =$\frac {((k+1)(k+2))}{2}$
\newline =RHS
\newline so, LHS = RHS for n = k +1, given that LHS = RHS for n = k 
\newline hence proved by induction
\end{itemize} 

\end{itemize} 

\begin{itemize}

\item $ 2 + 2^{2} + 2^{3} + 2^{4} + .... + 2^{n} = 2^{n+1} - 2 $

\begin{itemize}

\item Answer: 

for n = 1:

LHS = 2
\newline RHS = $2^{1+1} -2 = 2^{2} - 2 = 4 -2 = 2$
\newline so, LHS = RHS for n = 1
\newline Let's assume LHS = RHS for n = k:
\newline so, $2 + 2^{2} + 2^{3} + ... + 2^{k} = 2^{k+1} - 2 $
\newline for n = k + 1:
\newline LHS = so, $2 + 2^{2} + 2^{3} + ... + 2^{k} + 2^{k+1}$
\newline =$ 2^{k+1} - 2 + 2^{k+1}$
\newline =$ 2*(2^{k+1}) - 2$
\newline =$ 2^{k+2} -2 $
\newline =RHS
\newline so, LHS = RHS for n = k +1, given that LHS = RHS for n = k 
\newline hence proved by induction


\end{itemize} 

\end{itemize} 




\end{document}