\documentclass[11pt]{article}
\usepackage[margin=1in]{geometry}
\usepackage{amsmath}
\usepackage{enumitem}


\title{CSE - 015: Homework 3}
\author{Jaime Rivera}

\begin{document}
\maketitle



\section{Knights and Knaves:}
Recall the Knights and Knaves puzzle we saw in class. In that fictional world there are two types of people.
Knights - who always tell the truth, and Knaves - who always lie. It is impossible to distinguish them by
appearance, but only by the truth of their statements.

\begin{itemize}
\item One day a traveller was wandering around the island of Knights and Knaves, when he encountered
two local inhabitants, P and Q. The traveller asked: “Is any of you a knave?”. P replied: “At least
one of us is a knave”.

\begin{center}
\begin{tabular}{ |c|c|c|c| } 
 \hline
Case & P & Q \\ 
1 & Knight & Knight \\ 
2 & Knave & Knight \\ 
3 & Knight & Knave \\
4 & Knave & Knave \\
 \hline
\end{tabular}
\end{center}

\begin{itemize}
\item Can you find out what P and Q are? If so, what are they? If not, explain why not, and what other
information you would need to know. 
\begin{itemize}
\item Knave: always tells lie
\item Knight: always tells truth 
\item P's statement is false for the case 1 and True for the cases 2,3 and 4. If P is a Knight then what he says would be true, that means case 1 can't happen but case 3 can. If P is a Knave, whatever he would say must be False, meaning neither case 2 nor case 4 can happen as Knave always lie. Thus the only possible case is Row 3 and therefore P is a Knight and Q is a Knave.
\end{itemize}
\end{itemize}
\end{itemize} 
 
 
 
 
 \begin{itemize}
\item Later on, the traveller met two other locals, A and B. He asked whether either of them is a knight.
A replied: “If B is a knave, then I am a knave too”.

\begin{center}
\begin{tabular}{ |c|c|c|c| } 
 \hline
Case & P & Q \\ 
1 & Knight & Knight \\ 
2 & Knave & Knight \\ 
3 & Knight & Knave \\
4 & Knave & Knave \\
 \hline
\end{tabular}
\end{center}

\begin{itemize}
\item What are A and B?
 \begin{itemize}
\item Knave: always tells lie
\item Knight: always tells truth 
\item A's statement is false for the case 1 and True for the cases 2,3 and 4. If P is a Knight then what he says would be true, that means case 1 and case 3 can't happen. If P is a knave then what he says would be False, this means case 4 can't happen. The only possibility is that A is a Knave and B is a Knight, i.e, case 2 is the only possible scenario.
\end{itemize}
\end{itemize}
\end{itemize}

 



\section{Logical Identities: }

Simplify the following propositions. Show all steps of your solutions.

\begin{itemize}
\item $-(p \to (q \to p))$

 \begin{itemize}
\item According to the Implication Law: $p \to q \equiv \neg q \vee p $. so, $ \neg (p \to (\neg q \vee p) $
\item Again using Implication Law: $ \neg ( \neg p \vee (\neg q \vee p))$
\item According to De Morgans's Law: $\neg (a \vee b) \equiv \neg a \wedge \neg b,$  $\neg p \equiv  a $ $ and $ $(\neg q \vee p) \equiv b $ 
\begin{itemize}
\item $ \neg( \neg p) \wedge \neg ( \neg q \vee p)$
\end{itemize}
\item Again using De Morgans's Law: $\neg (a \vee b)$ $ \equiv $ $\neg a \wedge \neg b,$
\begin{itemize}
\item $\neg q $ is a 
\item p is b 
\item $ \neg (\neg p) \wedge \neg(\neg q) \wedge \neg p$
\end{itemize}
\item According to double negation law: $\neg (\neg p ) \equiv p $ 
\begin{itemize}
\item $p \wedge q \wedge \neg p $
\item $p \wedge \neg p \wedge q $
\end{itemize}
\item $p \wedge \neg p \equiv F $, thus 
\item $FALSE \wedge q  \equiv q $
\end{itemize}
\end{itemize}



 \begin{itemize}
\item $-p((p \wedge  q) \to (q \vee p))$

 \begin{itemize}
 \item Let $(p \wedge q)$ be a, $(q \vee p)$ be b 
\item According to Implication Law, $ p \to q \equiv \neg p \vee q $
 \begin{itemize}
 \item $ \neg a \vee b $
 \end{itemize}
\item Substituting the values of a and b we have,
 \begin{itemize}
 \item $ \neg (p \wedge q) \vee ( q \vee p)$ 
 \end{itemize}
\item According to De Morgan's Law: $ \neg(p \wedge q) \equiv \neg p \vee  \neg q$ 
 \begin{itemize}
 \item $\neg p \vee \neg q \vee (q \vee p)$ ----------- 1
\end{itemize}
\item According to associative law: $(p \vee q) \vee r \equiv p \vee (q \vee r) $
\item Rearranging 1, 
 \begin{itemize}
 \item $(\neg p \vee p) \vee (\neg q \vee q)$
\end{itemize}
\item According to negation laws, $ p \vee \neg p \equiv T$
\item $TRUE \vee TRUE$ $ \equiv TRUE$
\end{itemize}
\end{itemize}







\section{Logical Equivalences}

Determine whether or not the following pairs of propositions are equivalent. Show all steps.



\begin{itemize}
\item $ p \to (q \to r)$ and $(p \wedge q) \to r $

 \begin{itemize}
\item Answer: $ p \to (q \to r)$ $ \ne  $ $(p \wedge q) \to r $

\vskip 0.1in

\begin{tabular}{|c|c|c|c|c|}
\hline
$q$ & $r$ & $p$ & $q \to r$ & $p \to (q \to r)$ \\
\hline
0 & 0 & 0 & 1 & 1 \\
0 & 0 & 1 & 1 & 1 \\
0 & 1 & 0 & 1 & 1 \\
0 & 1 & 1 & 1 & 1 \\
1 & 0 & 0 & 0 & 1 \\
1 & 0 & 1 & 0 & 0 \\
1 & 1 & 0 & 1 & 1 \\
1 & 1 & 1 & 1 & 1 \\
\hline
\end{tabular}

\vskip 0.1in

and 

\vskip 0.1in

\begin{tabular}{|c|c|c|c|c|}
\hline
$p$ & $q$ & $r$ & $p \land q$ & $(p \land q) \to r$ \\
\hline
0 & 0 & 0 & 0 & 1 \\
0 & 0 & 1 & 0 & 1 \\
0 & 1 & 0 & 0 & 1 \\
0 & 1 & 1 & 0 & 1 \\
1 & 0 & 0 & 0 & 1 \\
1 & 0 & 1 & 0 & 1 \\
1 & 1 & 0 & 1 & 0 \\
1 & 1 & 1 & 1 & 1 \\
\hline
\end{tabular}



\end{itemize}

 \end{itemize} 







\begin{itemize}
\item $ p \to (q \to r)$ and $ (p \to q) \to r$

 \begin{itemize}
\item Answer: $ p \to (q \to r)$ $ \ne  $ $ (p \to q) \to r$

\vskip 0.1in

\begin{tabular}{|c|c|c|c|c|}
\hline
$q$ & $r$ & $p$ & $q \to r$ & $p \to (q \to r)$ \\
\hline
0 & 0 & 0 & 1 & 1 \\
0 & 0 & 1 & 1 & 1 \\
0 & 1 & 0 & 1 & 1 \\
0 & 1 & 1 & 1 & 1 \\
1 & 0 & 0 & 0 & 1 \\
1 & 0 & 1 & 0 & 0 \\
1 & 1 & 0 & 1 & 1 \\
1 & 1 & 1 & 1 & 1 \\
\hline
\end{tabular}

\vskip 0.1in
and
\vskip 0.1in

\begin{tabular}{|c|c|c|c|c|}
\hline
$p$ & $q$ & $r$ & $p \to q$ & $(p \to q) \to r$ \\
\hline
0 & 0 & 0 & 1 & 0 \\
0 & 0 & 1 & 1 & 1 \\
0 & 1 & 0 & 1 & 0 \\
0 & 1 & 1 & 1 & 1 \\
1 & 0 & 0 & 0 & 1 \\
1 & 0 & 1 & 0 & 1 \\
1 & 1 & 0 & 1 & 0 \\
1 & 1 & 1 & 1 & 1 \\
\hline
\end{tabular}





\end{itemize}

 \end{itemize} 







\section{Logical Consequence}



\begin{itemize}
\item 1. 

Jimmy is smart
\vskip 0.1in
Smart People are rich
\vskip 0.1in
 \textbf{----------------------------}
\vskip 0.1in
Jimmy is Rich 
 \vskip 0.1in
 
 
 \textbf{Answer:} Valid: Jimmy is rich 
 
 Jimmy is smart
\vskip 0.1in
Jimmy then smart
\vskip 0.1in
Jimmy $\to$ smart

Smart People are rich
\vskip 0.1in
Smart then rich 
\vskip 0.1in
Smart $\to$ rich 
\vskip 0.1in
Jimmy $\to$ rich $\equiv$ Jimmy is rich 

  


 \end{itemize} 
    
    
    
    
   
 
\begin{itemize}
\item 2. 

Islands are surrounded by water
\vskip 0.1in 
Puerto Rico is surrounded by water
\vskip 0.1in 
\textbf{--------------------------------------}
\vskip 0.1in 
Puerto Rico is an island
\vskip 0.1in


 \textbf{Answer:}  Not Valid: We cannot say Puerto Rico is an island is always true. 
 
 Islands are surrounded by water
 \vskip 0.1in
 Islands $\to$ surrounded by water
 \vskip 0.1in
  Puerto Rico is surrounded by water
 \vskip 0.1in
 Puerto Rico $\to$ surrounded by water
  \vskip 0.1in
  Puerto Rico is not always an island 
  
   
   
 \end{itemize} 






\end{document}