\documentclass[11pt]{article}
\usepackage[margin=1in]{geometry}
\usepackage{amsmath}
\usepackage{enumitem}


\title{CSE - 015: Homework 4}
\author{Jaime Rivera}

\begin{document}
\maketitle

\section{Mathematical Proofs:}

Prove (or disprove) the following results, showing all steps of your argument.

\begin{itemize}

\item 1. The sum of two odd integers is even.
\begin{itemize}
\item sum of two odd integers is even.
\item even numbers can be represented by = 2n.
\item odd numbers can be represented by = 2n+1.
\item sum of two odd numbers.
\item = $(2n+1) + (2n+2)$.
\item = $(4n+2) = 2(n+1)$.
\item It is multiples of 2, so it is divided by 2, therefore it is even. 
\item Even: [number divided by 2 and reminder will be '0'].
\item ODD: [number divided by 2 and reminder will be '1'].
\end{itemize} 

\item 2. The sum of two even integers is even.
\begin{itemize}
\item sum of two even integers is even
\item even number = 2n
\item sum of two even numbers = 2n+2n
\item =4n
\item =2(2n)
\item It is multiple of 2. So it is even 
\end{itemize} 


\item 3. The square of an even number is even.
\begin{itemize}
\item even number = 2n
\item square of even number = $(2n)(2n)$
\item = $4n^2$
\item = $2(2n^2)$
\item It is a multiple of 2, since it can be divided by 2 it is  even.
\end{itemize} 

\item 4. The product of two odd integers is odd.
\begin{itemize}
\item odd number = 2n+1
\item product of two odd numbers
\item = $(2n+1)(2n+1)$
\item = $4n^2 + 4n + 1$
\item = $2(2n^2 +2n) +1$ \textcircled{1}
\item If we divide \textcircled{1} then reminder is always (1) so it is odd.
\end{itemize} 


\item 5. If  $ n^3 +5 $ is odd then n is even, for any $n \in Z$
\begin{itemize}
\item Proof by contraposition
\item The contrapositive is “If n is odd, then n3 + 5 is even for any $n \in Z$ .” Assume that n is odd. We can now write n = 2k + 1 for some integer k. Then $n^{3} + 5 = (2k + 1)^{3} + 5 = 8k^{3} + 12k^{2} + 6k + 6 = 2(4k^{3} + 6k^{2} + 3k + 3)$. Thus $n^{3} + 5$ is two times some integer, so it is even by the definition of an even integer.
\end{itemize} 

\item 6. If $3n +2$ is even then n is even, for any $n \in Z$
\begin{itemize}
\item odd x odd = odd
\item odd + even = odd
\item Thus, if n was odd, then:
\item $3n+2$ = odd x odd + even = odd + even = odd
\item We know that $3n +2$ is even for any $n \in Z$
\item n can't be odd
\end{itemize} 


\item 7. The sum of a rational number and an irrational number is irrational.
\begin{itemize}
\item Proof by Contradiction
\item Assume:
\item rational + irrational = rational 
\item rational = a/b form 
\item $a/b + x = m/n$
\item $x = m/n - a/b$
\item $= \frac{mb - an}{nb}$ = it is in p/q form 
\item x is irrational (not in p/q form)
\item rational + irrational = rational 
\end{itemize} 


\item 8. The product of two irrational numbers is irrational.
\begin{itemize}
\item Proof:
\item two irrational numbers a,b
\item $a * b = c$
\item $1/n * n = 1 -> rational$
\item  $a * b = c$ 
\item $n*n = n^2 -> irrational$
\item Some times it is rational. Some time is is irrational. It depends on numbers $a,b$. 
\end{itemize} 

\end{itemize} 




\section{Basic Counting Principles:}

Answer the following questions. Show all steps of your solutions.

\begin{itemize}

\item 1. How many different three-letter initials can people have?
\begin{itemize}
\item In any three letter initial the word has to be among the 26 letter alphabet (a-z).
\item The total number of three letter  initials = $26 * 26 * 26 = 17576$.
\end{itemize} 

\item 2. How many different arrangements of the English alphabet are there?
\begin{itemize}
\item Since they are 26 letters in the English alphabet, so they can be arranged in 26! arrangements. 
\end{itemize} 


\item 3. There are 18 mathematics majors and 325 computer science majors at a college. In how many ways
can two representatives be picked so that one is a mathematics major and the other is a computer
science major?
\begin{itemize}
\item Select any one of the 18 students(mathematics majors) and any one of the 325 (computer science majors).
\item So total number of ways in which representatives can be selected is 18*325=5850.
\end{itemize}


\item 4. A particular brand of shirt comes in 12 colors, has a male version and a female version, and comes in
three sizes for each sex. How many different types of this shirt are made?
\begin{itemize}
\item Each sex total number of types of shirts $12*3=36$  for each color there are three sizes).
\item The total number of types of shirts for both sex will be $36*2=72$.
\end{itemize} 


\item 5. A multiple-choice test contains 10 questions. There are four possible answers for each question. In
how many ways can a student answer the questions on the test if the student answers every question?
\begin{itemize}
\item Each question can be answered in 4 ways so total number of ways in which all questions can be answered is
\item $4^{10}=1048576$
\end{itemize} 

\item 6. Suppose we have the same multiple choice test as described in question 5, but we relax the assumption
that the student has to answer all questions. In other words, how many ways are there for a student
answer the questions on the test if the student can leave answers blank?
\begin{itemize}
\item Let student answers i questions these i questions can be selected from 10 questions is 10Ci and after selecting these i questions can be answered in 4i ways so total number number of ways in which student can answer these i questions is....
\item $C_i * 4^{i}$ and i varies from 0 to 10 ......(student can select no question to answer or can answer all the questions)
\item Total number of ways will be equal to:

$(x + y)^{n} = \sum_{i=0}^{\infty}$  $C_{i}^{n} * x^{n-i} * y^{i}$ 
\vskip 0.1in 
$\sum_{i=0}^{i=10}$  $C_{i}^{10} * 4^{i} $ = $\sum_{i=0}^{i=10}$ $C_{i}^{10} * 4^{i} * 1^{10-i}$ = $(1+4)^{10} = 5^{10}$
\end{itemize} 
\end{itemize} 

\end{document}