\documentclass[11pt]{article}

\usepackage[margin=1in]{geometry}
\usepackage{amsmath}
\title{A Brief Article}
\author{Angelo Kyrilov}

\begin{document}

\maketitle

\section{Introduction}
Blah, blah, blah

\section{Other Stuff}

This is how you make a bullet list

\begin{itemize}
\item Point 1
\item Point 2
\end{itemize}

\noindent
and a numbered list

\begin{enumerate}
\item France
\item Croatia
\item Some other team
\end{enumerate}

\noindent
You can also make a table with borders, that is as big as the content:
\\\\
\begin{tabular}{|l|l|}
\hline
\textbf{Code} & \textbf{Name} \\
\hline
\hline
CSE15 & Discrete Mathematics\\
CSE30 & Data Structures \\
\hline
\end{tabular}
\\\\\\
\noindent
Or a custom sized table, without any borders:
\\\\
\begin{tabular}{p{2in} p{4.15in}}
    \textsc{Education} &  Some school\\\\
    & Some other school\\\\
    \textsc{Employment} & Some place\\\\
    & Some other place
\end{tabular}
\section{Mathematical Stuff}

One of the biggest advantages of \LaTeX~is its ability to produce beautifully typeset mathematics. Here is the quadratic formula:

$$x = -b \pm \frac{\sqrt{b^2-4ac}}{2a}$$

\noindent
You can also type math in the middle of paragraphs. For example the formula above is the solution to $ax^2+bx+c = 0$, which is the quadratic equation. You can even write fractions in paragraphs. Do you know what $\frac{1}{2} + \frac{1}{3}$ is equal to?

\LaTeX~offers everything you need to typeset mathematics, Greek letters, like $\alpha, \beta, \gamma, \pi, \theta, \varphi$, other useful symbols like $\infty, \sum, \in, \land, \lor,\lnot, \to$

You can also write solutions to equations, and it all lines up neatly. For example:

\begin{align*}
    x^2 + 2x - 2 &= 1\\
    x^2 + 2x - 3 &= 0\\
    (x + 3)(x - 1) &= 0\\
\end{align*}
Therefore $x = -3$ or $x = 1$

\end{document}