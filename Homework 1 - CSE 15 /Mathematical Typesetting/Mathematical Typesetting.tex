\documentclass[11pt]{article}

\usepackage[margin=1in]{geometry}
\usepackage{amsmath}



\title{Mathematical Typesetting}
\author{Jaime Rivera}

\begin{document}
 
\maketitle


\begin{enumerate}

\item (x-2)(x+4) =7

\begin{align*}
    (x-2)(x+4) =7\\
    x^2 +2x -8 = 7\\
   x^2+2x-8-7=0\\
   x^2+2x-15=0\\
   (x-3)(x+5)=0\\
    \end{align*}
Therefore, $x = 3$ or $x = -5$


\item $(x-3)^2=4$

  \begin{align*}
 (x-3)^2=4\\
 (x-3)={\sqrt4}\\
 (x-3) = \pm2 \\
 x = 3 - 2\\
 x= 3 + 2\\
  \end{align*}
 
 Therefore, $x = 5$ or $x = 1$
 

\item $\sqrt{(x-2)} + 2 = 4$
 
  \begin{align*}
 \sqrt{(x-2)} + 2 = 4\\
 \sqrt{(x-2)} = 4 - 2\\
\sqrt{(x-2)} = 2\\
\sqrt{(x-2)}^2 = (2)^2\\
x-2 = (2)^2\\
x - 2 = 4\\
  \end{align*}
 
 Therefore, $x = 6$


\end{enumerate}

 
 
\end{document}